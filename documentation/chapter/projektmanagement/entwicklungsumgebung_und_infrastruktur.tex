\section{Entwicklungsumgebung und Infrastruktur}
\subsection{IDE (Integrated Development Environment)}
\decision{PyCharm}	
Beiden Projektmitgliedern ist JetBrains Intellij bekannt und \Gls{PyCharm} ist im Umgang nahe zu identisch.
Für Studenten sind die Entwicklungsumgebungen kostenlos verfügbar.
\subsection{SCM (Source Control Management)}
\decision{GitHub}
Der Umgang mit \Gls{Git} ist beiden Projektmitglieder bestens bekannt.
\Gls{GitHub} ist ohne Unkosten von überall verfügbar.
Das Geometalab der HSR publiziert über diesen Weg diverse Projekte.

\subsection{CI (Continuous Integration)}
\decision{CircleCI}
Das finden eines passenden Continuous Integration Tools stellte sich schwieriger dar, als zu Beginn des Projektes erwartet. Während dem SE2 Projekt haben wir Bekanntschaft mit Travis CI gemacht, welches die vielen Abhängigkeiten unseres Codes nicht abdecken konnte. Mit CircleCI fanden wir eine Lösung, die auf Docker Hub zugreifen kann, dann den Build des Images durchführt und schlussendlich die Test durchführt.

\subsection{Projektmanagement Tool}
\decision{Jira}
\Gls{Jira} ist den Projektmitgliedern schon aus dem SE2-Projekt bekannt und hat sich sehr bewährt.
Das Dashboard ist übersichtlich gestaltet. Es ermöglicht eine Übersicht über die aktuellen Tasks auf einen Blick.
Alle Mitglieder haben jederzeit Zugriff auf die Plattform, was die Transparenz erhöht.
Weiter bietet Jira diverse Reports um Auswertungen über das Projekt zu fahren.
