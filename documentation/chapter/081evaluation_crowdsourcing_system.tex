\section{Evaluation Crowdsourcing-System}
\subsection{Kandidaten}
\begin{itemize}
	\item MapRoulette\footnote{\url{http://maproulette.org/}} 
	\item To-Fix\footnote{\url{http://osmlab.github.io/to-fix/\#/task/tigerdelta}}
\end{itemize}

\subsection{MapRoulette}
MapRoulette verwendet für ihre Challenges und Tasks ein einfaches JSON Format. Der erstellt werden Challenges mittels POST und mit PUT können diese upgedatet werden.

\subsection*{Beispiel Challenge}
\begin{tabbing}
    \hspace*{2cm}\=\hspace*{3cm}\= \kill
    Erstellen: \> POST /api/admin/challenge/<slug>  \\
    Updaten: \> PUT /api/admin/challenge/<slug> \\
    Challenge JSON: \\
\end{tabbing}		
\begin{python}
{
  "title": "Repair Motorways",
  "description": "Repair all motorways",
  "blurb": "For this challenge, the idea is to repair all motorways",
  "help": "Repair the motorway where it is broken as indicated on the map",
  "instruction": "Look at the map for broken pieces of way.",
  "active": true,
  "difficulty": 2
}
\end{python}